% !TeX root = ../libro.tex
% !TeX encoding = utf8
%
%*******************************************************
% Summary
%*******************************************************

\selectlanguage{english}
\chapter{Abstract}

\noindent The main purpose of the work detailed below is to show a possible mathematic modelization of a Convolutional Neural Network and demonstrate one of the main properties of this kind of network, Translation Invariance.  On the other side, we will adapt the architecture of an existing network in order to solve a real problem about cephalometric landmark detection from a Machine Learning perspective using few-shot learning.

\medskip

\noindent Convolutional Neural Networks are recent tools which have demonstrated a great capacity for image-processing tasks. Their good performance on this kind of tasks can be checked empirically. However, their mathematical modelization and the theoretical justification of why they are good for this kind of tasks still is an open case study. For this reason, the main purpose of this part of the work is to present a possible modelization for CNN based on the theoretical approximation presented by Mallat in his work \textbf{Group Invariant Scattering}. Once the modelization is shown, we try to prove one of the most important properties of this network, Translation Invariance.

\medskip

\noindent To do so, we present an operator called scattering propagator. Using this operator in a cascade of convolutions we reach the windowed scattering transform, presented as the mathematical modelization of CNN. We present a set of properties that this operator must satisfy, like Lipschitz-continuity, non-expansivity and translation invariant coefficients. The operator that satisfies all these properties is the Littlewood-Paley wavelet transform. Once we have the operator we define the modelization of the windowed scattering and we study the similarities with the Convolutional Neural Networks. To prove the translation invariance we use that the windowed operator is non-expansive.

\medskip

\noindent In the second part of this work, we adapt an existing CNN specialized in facial landmarks detection called 3FabRec (which is an Adversarial Autoencoder with interleaved layers that predict landmarks). The objective is to predict cephalometric facial landmarks using this framework. Cephalometric landmarks have biological inspiration. The main problem is the small dataset provided with only a few images in-the-wild to train the model.

\medskip

\noindent To do so, the dataset provided had 167 images of people in the wild. We wanted to train the framework to predict a maximum of $30$ landmarks in each image. After a previous analysis of the dataset, we discovered that not all the landmarks are marked in all images, and we needed to use an auxiliary network called Facenet to identify bounding boxes on the images and be able to train the model. Due to the few amount of data we have, this is contained in the field of few-shot learning. 

\medskip

\noindent We studied the state-of-art discovering that only a few articles in the search we did were related to this specific problem. In these articles, a set of images in the same position and illumination conditions were used, instead of the in-the-wild dataset we use. Our model solves the problem using a Network pretrained on the AFLW dataset and making fine-tuning with the forensic dataset. We apply data augmentation techniques to do so and we obtain competitive results like an average NME of $2.65$. 

\medskip

\noindent Finally, we compare our results with the model proposed by Guillermo Gómez in his project on $2019$. The median RMSE of our model is $5.3862$ and in the model presented in $2019$ is $3.4106$. Despite the lower average RMSE error, our model improves the performance in some landmarks using a smaller dataset.

% Al finalizar el resumen en inglés, volvemos a seleccionar el idioma español para el documento
\selectlanguage{spanish} 
\endinput
