% !TeX root = ../libro.tex
% !TeX encoding = utf8
%
%*******************************************************
% Summary
%*******************************************************

\selectlanguage{english}
\chapter{Abstract}

\noindent The main purpose of the work detailed below is to show a possible mathematic modelization of a Convolutional Neural Network and demonstrate one of the main properties of this kind of network, Translation Invariance.  On the other side, we will adapt the architecture of an existing network in order to solve a real problem.

\medskip

\noindent Convolutional Neural Networks are recent tools which have demonstrated a great capacity for image-processing tasks. Their good performance on this kind of task can be checked empirically. However, their mathematical modelization and the theoretical justification of why they are good for this kind of task still is an open case study. For this reason, the main purpose of this part of the work is to present a possible modelization for Convnets based on the theoretical approximation presented by Stephane Mallat in his work Group Invariant Scattering. Once the modelization is shown, we try to prove one of the most important properties of this network, Translation Invariance. The main difficulty is that we do not have many articles to compare related information, and many times, the way are presented the main results may sound confusing, that is the reason why we try to explain these steps with clarity.

\medskip

\noindent In this work appears concepts of functional analytics, Fourier analysis and signal processing via wavelets.  A great part of the process consisted of the study and investigation of new concepts like wavelets. The principal book consulted was Digital image processing by Rafael Gonzalez.

\medskip

\noindent First, we present an operator called scattering propagator. Using this operator in a cascade of convolutions we reach the windowed scattering transform, presented like the mathematical modelization of Convnets. We present a set of properties that this operator must verify, Lipschitz-continuity, like non-expansivity and translation invariant coefficients. The operator that verifies all these properties is the Littlewood-Paley wavelet transform. Once we have the operator we define the modelization of the windowed scattering and we study the similarities with the Convolutional Neural Networks. 





% Al finalizar el resumen en inglés, volvemos a seleccionar el idioma español para el documento
\selectlanguage{spanish} 
\endinput
