% !TeX root = ../libro.tex
% !TeX encoding = utf8
%
%*******************************************************
% Introducción
%*******************************************************

% \manualmark
% \markboth{\textsc{Introducción}}{\textsc{Introducción}} 

\chapter{Resumen}

%Introducción a ambos trabajos
\noindent El trabajo que a continuación se detalla tiene como objetivo presentar una posible modelización, desde el punto de vista matemático, de lo que es una Red Neuronal Convolucional junto con la demostración de una de sus principales propiedades: la invarianza frente a traslaciones. Por otro lado, se pretende adaptar una arquitectura de red neuronal convolucional ya existente a un problema real que resulte apropiado para este tipo de algorimos, que se abordará desde el punto de vista informático.

\medskip

%Se detalla la tarea de la primera parte así como sus problemas
\noindent En lo que respecta a la primera parte del trabajo, las \textit{Convnets} son herramientas que pese a haber surgido recientemente han demostrado tener un gran potencial para el procesamiento de imágenes, convirtiéndose rápidamente en una de las principales herramientas para estos fines. Su buen rendimiento en este tipo de tareas es comprobable empíricamente, sin embargo siguen siendo una vía de estudio abierta en lo que se refiere a la modelización matemática que siguen y la justificación teórica de su buen comportamiento. Es por ello que en el presente trabajo nos proponemos tratar de explicar al lector de la forma más clara posible una posible modelización matemática propuesta por Stephane Mallat en su trabajo \textbf{Group Invariant Scattering}, que será la principal fuente consultada. Así mismo, una vez propuesta la modelización, se pretende demostrar una de las principales propiedades conocidas de las redes neuronales convolucionales, que es la invarianza por traslaciones. La principal dificulad reside en el hecho de que no existe mucha bibliografía al respecto con la que contrastar la información y en ocasiones, los resultados y conclusiones escritas por el autor pueden ser difíciles de entender. Es ahí dónde entra este trabajo, que se propone arrojar luz en todos esos procedimientos que pueden resultar confusos.

\medskip
%Se detallan el enfoque elegido a seguir y por qué 
\noindent Los principales conceptos tratados son propios del análisis funcional, análisis de Fourier y procesamiento de señales por medio de ondeletas. Una parte importante del trabajo consistió en un gran periodo de investigación y consulta de diversos libros y artículos, especialmente para comprender el concepto de ondeleta, del cual la principal fuente ha sido el libro de Mallat \textbf{A Wavelet Tour of Signal Processing} y el estudio del procesamiento de imágenes por medio de transformadas de ondeletas destacando el libro \textbf{Digital image processing} de Rafael González. Los conocimientos adquiridos en esta etapa fueron fundamentales para el posterior desarrollo trabajo.


\medskip

\noindent En lo que respecta a la modelización de las redes neuronales convolucionales comenzamos con la búsqueda de un operador que denominaremos \textit{propagador de dispersión}, el cual al aplicarlo en cascada usando la operación de convolución nos de como resultado el denominado \textit{operador de ventana}, que será la modelización propuesta y que actuará sobre caminos de frecuencias finitos (pues como demostraremos, con un camino de frecuencias finito conseguimos retener tanta información como se quiera). En primer lugar se definen un conjunto de características deseables que tendría que cumplir este operador como la \textit{Lipschitz-continuidad}, que sea no expansivo o que produzca coeficientes invariantes a traslaciones. Con todo esto, se demuestra que no podemos usar herramientas clásicas como son la transformada de Fourier, y debemos recurrir a la transformada de ondeletas de \textit{Littlewood-Paley} para encontrar el operador que cumple con todos los requisitos. Una vez encontrado, describiremos las diferencias y similitudes que se aprecian entre el modelo propuesto y la arquitectura de una red neuronal convolucional clásica. 

\medskip

\noindent En lo que respecta a la invarianza por traslaciones, haciendo uso de las propiedades que le hemos exigido al \textit{operador de ventana}, se consigue demostrar en primer lugar que dicho operador es no expansivo al aplicarse en conjuntos de caminos de frecuencias, y con dicha propiedad se demuestra la invarianza por traslaciones del mismo.

\medskip
%Se resumen las conclusiones extraidas
\noindent Como resultado del proceso, se concluye que finalmente se cumplen todos los objetivos propuestos aunque el desarrollo del trabajo ha sido complicado debido al poco detalle presente en algunos resultados,o debido a que los mismos eran demasiado técnicos. No obstante, los resultados obtenidos suponen una aproximación fiel a las propiedades que poseen las redes neuronales convolucionales y se proporcionan justificaciones para las mismas.

\medskip

%Se detalla la tarea de la segunda parte así como sus problemas.
\noindent En la segunda parte del trabajo, se pretende adaptar un framework existente especializado en el reconocimiento de landmarks faciales denominado \textbf{3FabRec} (que es un \textit{Adversarial Autoencoder} con capas auxiliares encargadas del reconocimiento de landmarks en imágenes), para que sea capaz de marcar landmarks cefalométricos (puntos antropométricos situados en la cabeza) empleando un conjunto de datos reducido y con imágenes en diversas posiciones y condiciones de calidad e iluminación. Los principales inconvenientes a la hora de realizar esta parte son el pequeño conjunto de datos con el que se cuenta tanto para entrenar como para hacer validación, así como la necesidad de recortar cada imagen para quedarnos con el rostro de cada sujeto. Se realizaŕa un ajuste fino de la red, aprovechando su conocimeinto previo en el marcado de landmarks no morfológicos, intentando que aprenda a identificar landmarks cefalométricos. La resolución de este problema se encuentra dentro del ámbito de la Antropología forense.

\medskip

%Se explican el conjunto de datos del que se parte y los métodos usados.
\noindent Para llevar a cabo esta tarea se parte de un conjunto inicial de 167 imágenes de distintos sujetos en una gran variedad de ámbitos, posturas, condiciones de iluminación y de calidad de imagen con los landmarks etiquetados por un experto, este conjunto de datos se dividirá en entrenamiento y test. Nuestra tarea será la de entrenar la red neuronal mencionada anteriormente para lograr predecir hasta un máximo de \textbf{30 landmarks} distribuidos por toda la cara en una imagen reconstruida del rostro de entrada. Tras un primer análisis vemos como no todos los landmarks se encuentran marcados en todas las imágenes del dataset. Por otro lado, usamos una red neuronal auxiliar para la identificación y recorte de rostros en la base de datos, aunque vemos que en ciertas imágenes de perfil no realiza la identificación correctamente, por lo que se propone extraer dichas imágenes de la base de datos que se usará para el resto del estudio. 

\medskip

%Se explica el estado del arte y por qué es una apuesta novedosa y ver si constituye el estado del arte en este campo.
\noindent Además, se ha revisado con detenimiento el estado del arte en este campo, descubriendo que se trata de una aproximación novedosa pues los trabajos encontrados que guarden relación sobre el tema han sido en total cuatro, uno de ellos guarda relación con un problema estrechamente relacionado con el que trataremos que es el marcado de landmarks craneométricos, los otros tres guardan relación directa con el marcado de landmarks cefaloméricos.En uno de ellos trabajan con un número pequeño de landmarks y aplicando técnicas que no pertenecen al ámbito del \textit{deep learning}, en los otros, aunque utilizan una red neuronal convolucional, entrenan el modelo con imágenes tomadas con buena calidad, iluminación y sin variaciones de posición.

%Se describen los experimentos y los resultados
\medskip

\noindent El entrenamiento se realiza a partir del conocimiento previo en el dataset \textit{AFLW} con los nuevos datos siguiendo diversos enfoques. En primer lugar se establece un modelo base en el que simplemente entrenamos las capas de la red relacionadas con el marcado de landmarks, tras esto obtenemos buenos resultados pero vemos cómo en algunos casos las reconstrucciones llevadas a cabo por el \textit{Adversarial Autoencoder} no son muy realistas, en especial en las imágenes de perfil, por lo que se adoptarán dos enfoques para tratar de mejorar el marcado: El primero consiste en reentrenar partes de la red encargadas para mejorar la reconstrucción, y el segundo entrenar en un dataset mayor aplicando técnicas de data augmentation y que complique los ejemplos de entreamiento. 

\medskip 

\noindent Para el primer enfoque se realizan dos experimentos, el primero es reentrenar las capas encargadas del marcado de landmarks junto con el encoder (dejando congelados los pesos del decoder para prevenir overfitting), y la segunda alternativa probada sería la contraria, entrenar decoder y capas de marcado dejando fijos los pesos del encoder. Ambas pruebas no mejoran excesivamente el marcado de landmarks y apenas reducen el error de reconstrucción, además de que son más costosas computacionalmete por lo que se opta por otra vía. 

\medskip

\noindent En la segunda alternativa se amplían las imágenes de entrenamiento introduciendo imágenes rotadas, trasladadas, giradas y con oclusiones parciales. Esto incrementa la dificulad pero como vereos los resultados obtenidos son muy satisfactorios tanto en validación como en el conjunto de test. Se concluye así que el método obtiene muy buenos resultados en el problema que se propone solucionar y se abren nuevas vías de investigación futuras.

\endinput
