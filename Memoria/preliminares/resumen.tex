% !TeX root = ../libro.tex
% !TeX encoding = utf8
%
%*******************************************************
% Introducción
%*******************************************************

% \manualmark
% \markboth{\textsc{Introducción}}{\textsc{Introducción}} 

\chapter{Resumen}

%Introducción a ambos trabajos
\noindent El trabajo fin de grado (TFG) que a continuación se detalla tiene como objetivo presentar una posible modelización, desde el punto de vista matemático, de lo que es una \textbf{Red Neuronal Convolucional} (CNN), junto con la demostración de una de sus principales propiedades: la invarianza frente a traslaciones. Por otro lado, se pretende adaptar una arquitectura de red neuronal convolucional ya existente a un problema real que resulte apropiado con \textit{few-shot} learning, que se abordará desde el punto de vista informático.

\medskip

%Se detalla la tarea de la primera parte así como sus problemas
\noindent Las CNN pese a haber surgido recientemente, han demostrado tener un gran potencial para el procesamiento de imágenes, convirtiéndose rápidamente en una de las principales herramientas para estos fines. Su buen rendimiento en este tipo de tareas es comprobable empíricamente. Sin embargo, siguen siendo una vía de estudio abierta en lo que se refiere a la modelización matemática y la justificación teórica de estos resultados. Así, en la primera parte del TFG se pretende desarrollar \textbf{la modelización matemática} propuesta por Mallat, en su trabajo \textbf{Group Invariant Scattering}, junto con la demostración de una de sus principales propiedades: \textbf{la invarianza por traslaciones}.

\medskip

\noindent En lo que respecta a la modelización, comenzamos con la búsqueda de un operador denominado \textit{propagador de dispersión}, el cual,  aplicando la operación de convolución de forma recursiva, construye el denominado \textit{operador de ventana}, presentado como la modelización matemática de una CNN, y actuará sobre caminos de frecuencias finitos (pues, como demostraremos, se puede conseguir retener tanta información como se quiera). Algunas propiedades importantes que debe verificar nuestro operador son la \textit{Lipschitz-continuidad}, que sea no expansivo o que produzca coeficientes invariantes a traslaciones. Dichas propiedades nos impedirá usar herramientas clásicas como la transformada de Fourier, y tendremos que recurrir a la transformada de ondeletas de \textit{Littlewood-Paley}. A continuación, se realiza un estudio de las diferencias y similitudes que guarda con las CNN tal y como las conocemos.

\medskip

\noindent En lo que respecta a la invarianza por traslaciones, haciendo uso de las propiedades del \textit{operador de ventana}, se consigue demostrar, en primer lugar, que dicho operador es no expansivo al aplicarse en conjuntos de caminos de frecuencias, y con dicha propiedad se demuestra su invarianza por traslaciones.

\medskip

%Se detalla la tarea de la segunda parte así como sus problemas.
\noindent En la segunda parte del trabajo, se pretende adaptar un framework, ya existente, especializado en el reconocimiento de landmarks faciales, denominado \textbf{3FabRec} (un \textit{Adversarial Autoencoder} con capas auxiliares encargadas del reconocimiento de landmarks en imágenes) para que sea capaz de marcar landmarks cefalométricos, puntos antropométricos situados en la cabeza, empleando un conjunto de datos reducido de imágenes en diversas posiciones y condiciones de calidad e iluminación. Además, se modificará su función de coste para poder ser entrenado con landmarks faltantes.

\medskip

%Se explican el conjunto de datos del que se parte y los métodos usados.
\noindent Para llevar a cabo esta tarea se parte de un conjunto de 167 imágenes de distintos sujetos en una gran variedad de posturas, condiciones de iluminación y de calidad de imagen con los landmarks etiquetados por un experto. Este conjunto de datos se dividirá en entrenamiento y test. Entrenaremos la red neuronal mencionada anteriormente para lograr predecir hasta un máximo de \textbf{30 landmarks} distribuidos por toda la cara. Tras un análisis previo de los datos, detectamos la presencia de landmarks faltantes en la mayoría de imágenes, así como la necesidad de emplear un identificador de caras para marcar \textit{bounding boxes} en las imágenes y poder recortarlas en un paso previo al entrenamiento de la red. Debido a los pocos ejemplos de entrenamiento que poseemos, consideramos el problema dentro del ámbito del \textit{few-shot learning}.

\medskip

%Se explica el estado del arte y por qué es una apuesta novedosa y ver si constituye el estado del arte en este campo.
\noindent Los trabajos encontrados que guardan relación sobre el tema son muy escasos, y los que aplican técnicas de deep learning, no emplean conjuntos de datos \textit{in-the-wild}. Nuestra propuesta plantea aprovechar el conocimineto adquirido por una red en un amplio dataset, como es \textit{AFLW}, en el reconocimiento de landmarks no antropométricos y ajustarla mediante \textit{fine-tuning}  y técnicas \textit{data augmentation}  para el reconocimiento de landmarks cefalométricos. Vamos a resolver, por tanto, un problema de \textbf{regresión} y, como veremos, los resultados obtenidos son competitivos con el estado del arte. La media del RMSE cometido por nuestro modelo es de $2.7$ píxeles de error, frente a los $3.41$ del mejor modelo. Esta mejora a nivel global en el marcado también se estudia a nivel de landmark.

\endinput
