\chapter{Métricas}

    \noindent En este capítulo se van a presentar las principales métricas de error que se emplearán para estudiar la bondad de los resultados que se obtengan en futuros capítulos.

    \section{Métricas usadas en el entrenamiento}
        \noindent Las métricas que se muestran durante el entrenamiento utilizadas por el modelo son:

        \subsection{MSE}
            \noindent Se trata del error cuadrático medio, una función empleada típicamente en problemas de regresión y que tiene la siguiente expresión: 

            \begin{equation}
                MSE = \frac{1}{N} \sum_{i=1}^{N} (y_i - \widehat{y}_i)^2
            \end{equation}

            \noindent En nuestro caso concreto, este error se calcula entre los mapas de calor de los landmarks reales y predichos a nivel de pixel, es decir $y_i$ en la fórmula anterior corresponde a cada pixel del heatmap del landmark real y por tanto $\widehat{y}_i$ es el valor del pixel correspondiente en el heatmap predicho. Dichas diferencias se suman en todos los heatmaps y se divide por $N$ que en este caso es el número total de píxeles juntando todos los Heatmaps.

            \medskip

            \noindent Este error se empleará para realizar \textit{backpropagation} a través de las ITLs de la red y será el responsable del aprendizaje de la red.

        \section{Métricas empleadas en validación y testing}

        \subsection{Error de reconstrucción}
            \noindent Aunque no se trata de un métrica propia de un problema de regresión, el valor del error de reconstrucción en las imágenes de validación nos llevará en la sección de experimentación a realizar hipótesis sobre la relación entre este error y el NME.

            \medskip

            \noindent Se trata de la función de coste \textbf{L1} que se aplica a la imagen original y la reconstruida  y nos porporciona una medida del error de reconstrucción a nivel de pixel. Su expresión es la siguiente:

            \begin{equation}
                Reconstruction \; Loss = \sum_{i=1}^n |p_i -q_i|
            \end{equation}

            \noindent Dónde $p_i$ representa cada pixel de la imagen reconstruida por la red y $q_i$ el equivalente en la imagen real.

        \subsection{Normalized Mean Error}
            \noindent Este será el error utilizado para medir la bondad de los resultados. Tiene la siguiente expresión para cada landmark de los $30$: 

            \begin{equation}
                NME=\frac{\sqrt{\sum_{i=1}^{n}(p_{i,1} -q_{i,1})^2+ (p_{i,2} -q_{i,2})^2}}{N}
            \end{equation}


            \noindent Así, si en total se han identificado en las imágenes de validación un total de $n$ landmarks de cierto tipo en las imágenes de test o validación, se calcula la distancia $L_2$ entre el landmark de la imagen $i$ predicho y el real. Y tras esto se suman todos los valores obtenidos. Como podemos ver, idealmente $n$ debería ser el total de imágenes que hay en validación o test, pero no todos los landmarks aparecen en todas las imágenes, por lo tanto $n\leq M$ siendo $M$ el total de imágenes en validación.

            \noindent En dicha expresión $N$ es el término por el que se normaliza, y varía según el problema. En nnuestro caso normalizaremos la media del numerador por el ancho de la imagen que es de $256$ píxeles, ya que la distancia máxima permitida entre dos landmarks sería esta, y así todos los valores quedarían entre $0$ y $1$ siendo $0$ una estimación perfecta del landmark, y $1$ un error total en el marcado.

            \subsection{SSIM}
            \noindent Se trata del \textit{structural similarity index} (SSIM) y da una medida de \textit{similaridad} entre dos imágenes \cite{wang2004image}, su expresión es la siguiente: 

            \begin{equation}
                SSIM(x,y)=[l(x,y)]^\alpha[c(x,y)]^{\beta}[s(x,y)]^{\gamma}
            \end{equation}

            \medskip

            \noindent Dónde $x,y$ son las dos imágenes que van a ser comparadas.

            \medskip

            \noindent La componente $c(x,y)$ hace referencia a la función de comparación del contraste de las dos imágenes y viene dada por la siguiente expresión: 

            \begin{equation}
                c(x,y)=\frac{2\sigma_x \sigma_y + C}{\sigma_x^2+ \sigma_y^2+C}
            \end{equation}

            \noindent Dónde $\sigma_x, \sigma_y$ hacen referencia a la desviación estándar de cada imagen.

            \medskip

            \noindent La componente $s(x,y)$ es la función de comparación estructural entre las dos imágenes y viene dada por la siguiente expresión:

            \begin{equation}
                s(x,y)=\frac{\sigma_{xy}+C/2}{\sigma_x \sigma_y +C/2}
            \end{equation}

            \noindent Dónde $\sigma_{xy}$ denota la covarianza.

            \medskip

            \noindent La componente $l(x,y)$ hace referencia a la \textit{luminosidad}, pero en el caso de 3FabRec se prescinde de esta componente, además los exponentes $\alpha, \beta , \gamma$ se igualan a $1$.

            \medskip

            \noindent Por otra parte siguen las indicaciones del paper original de SSIM que recomiendan usar estas comparaciones en regiones de la imagen y promediarlas en vez de aplicarlas sobre todo el conjunto de píxeles de la imagen, es por ello que la expresión final queda:

            \begin{equation}
                cs(x,y)=\frac{1}{|w|} \sum{c(x_w,y_w)s(x_w,y_w)}_w
            \end{equation}

            \noindent Dónde $w$ representa la ventana sobre la que se aplica la función y $|w|$ el total de ventanas. En nuestro caso se emplean ventanas de tamaño $31\times 31$.




\endinput
%------------------------------------------------------------------------------------
% FIN DEL CAPÍTULO. 
%------------------------------------------------------------------------------------

