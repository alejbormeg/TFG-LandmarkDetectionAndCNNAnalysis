\chapter{Datos y Métricas}

    \noindent En este capítulo se van a presentar los datos de los que disponemos para resolver el problema principal de este trabajo (la identificación de landmarks cefalométricos mediante técnicas de few-shot learning), así como las principales métricas de error que se emplearán para estudiar la bondad de los resultados que se obtengan en futuros capítulos.

    \section{Metricas}
        
        \noindent A conitnuación mostraremos las métricas empleadas para estudiar la bondad de los resultados. Algunas de estas métricas ya se emplearon en la funciones de coste del apartado anterior. 

        \subsection{SSIM}
            \noindent Se trata del \textit{structural similarity index} (SSIM) y da una medida de \textit{similaridad} entre dos imágenes \cite{wang2004image}, su expresión es la siguiente: 

            \begin{equation}
                SSIM(x,y)=[l(x,y)]^\alpha[c(x,y)]^{\beta}[s(x,y)]^{\gamma}
            \end{equation}

            \medskip

            \noindent Dónde $x,y$ son las dos imágenes que van a ser comparadas.

            \medskip

            \noindent La componente $c(x,y)$ hace referencia a la función de comparación del contraste de las dos imágenes y viene dada por la siguiente expresión: 

            \begin{equation}
                c(x,y)=\frac{2\sigma_x \sigma_y + C}{\sigma_x^2+ \sigma_y^2+C}
            \end{equation}

            \noindent Dónde $\sigma_x, \sigma_y$ hacen referencia a la desviación estándar de cada imagen.

            \medskip

            \noindent La componente $s(x,y)$ es la función de comparación estructural entre las dos imágenes y viene dada por la siguiente expresión:

            \begin{equation}
                s(x,y)=\frac{\sigma_{xy}+C/2}{\sigma_x \sigma_y +C/2}
            \end{equation}

            \noindent Dónde $\sigma_{xy}$ denota la covarianza.

            \medskip

            \noindent La componente $l(x,y)$ hace referencia a la \textit{luminosidad}, pero en el caso de 3FabRec se prescinde de esta componente, además los exponentes $\alpha, \beta , \gamma$ se igualan a $1$.

            \medskip

            \noindent Por otra parte siguen las indicaciones del paper original de SSIM que recomiendan usar estas comparaciones en regiones de la imagen y promediarlas en vez de aplicarlas sobre todo el conjunto de píxeles de la imagen, es por ello que la expresión final queda:

            \begin{equation}
                cs(x,y)=\frac{1}{|w|} \sum{c(x_w,y_w)s(x_w,y_w)}_w
            \end{equation}

            \noindent Dónde $w$ representa la ventana sobre la que se aplica la función y $|w|$ el total de ventanas. En nuestro caso se emplean ventanas de tamaño $31\times 31$.

        \subsection{Average pixel error}
            \noindent Se trata de la función de coste \textbf{L1} que se aplica a la imagen original y la reconstruida  y nos porporciona una medida del error de reconstrucción a nivel de pixel. Este error se obtiene para cada imagen y luego se devuelve el error promedio, por eso se denomina \textit{Average pixel error}.

        \subsection{MSE}
            \noindent Se trata de la función de coste \textbf{L2} usada para medir la diferencia entre los mapas de calor pertenecientes a los landmarks reales y los mapas de calor de los landmarks predichos.Su expresión es \eqref{eq::L2}


\endinput
%------------------------------------------------------------------------------------
% FIN DEL CAPÍTULO. 
%------------------------------------------------------------------------------------

