\chapter{Conclusiones y Trabajos Futuros}

\section{Objetivos Satisfechos}

Todos los objetivos que se habían propuesto al comienzo de este trabajo se han visto realizados con éxito:

\begin{enumerate}
    \item Se ha realizado una minuciosa investigación sobre el \textbf{estado del arte} en el campo concluyendo que estamos en una vía de estudio de la que apenas hay artículos publicados. Además, hemos resumido y estudiado todas las propuestas de los pocos trabajos encontrados relacionados con el problema de forma directa, con el fin de conocer qué vías se han explorado para su resolución con anterioridad.
    \item Hemos profundizado en el conocimiento de las \textbf{\textit{redes neuronales convolucionales}}, y en técnicas de entrenamiento \textit{few-shot} y con \textit{data augmentation}. Además nos hemos familiarizado con la librería \textit{pytorch}, una de las más empleadas actualmente en este tipo de tareas.
    \item Hemos realizado un estudio para observar la \textbf{evolución} de los Autoencoders y las redes Adversarias, comenzando por describir lo que es un Autoencoder clásico, las novedades que incorporan las GANs, la aparición de los VAE, y finalmente la combinación de ambos en la arquitectura del AAE.
    \item Se ha realizado un \textbf{estudio de la base de datos} proporcionada viendo la frecuencia de aparición de landmarks en cada imagen, así como el rendimiento del detector de caras sobre las imágenes del dataset agrupadas por frontales, de perfil o $3/4$. Con esto se pudo hacer una estimación de los landmarks que serían más difíciles de predecir y se descartaron tres imágenes debido a que el detector de caras no pudo reconocer ninguna en las mismas.
    \item Con los resultados del anterior estudio se construyó un \textbf{nuevo dataset} compuesto por las imágenes usadas finalmente junto con los \textit{bounding boxes} asociados a fin de que el framework realizara el cropping sobre las imágenes antes de ser usadas por la red.
    \item Finalmente se realizó un \textbf{estudio experimental} en el cual se probaron diversas técnicas para predecir los landmarks. En primer lugar se optó por entrenar las ITLs de la red a partir del conocimiento previo de la red en el dataset \textit{AFLW} (elegido por compartir características con el nuestro), se vieron que los resultados eran buenos pero los errores de reconstrucción altos. Tras esto se optó por tratar de reducir los errores de reconstrucción a ver si así mejoraba el marcado de landmarks, se entrenaron las ITLs de la red junto con el \textit{encoder} dejando congelados los pesos del \textit{decoder} y luego se repitió el mismo experimento pero entrenando el \textit{decoder} y congelando el \textit{encoder}. Los resultados de estos experimentos no fueron satisfactorios pues el error de reconstrucción apenas bajó y el marcado de landmarks no mejoró. Finalmente se optó por aplicar técnicas de \textit{Data augmentation} sobre el dataset original. Realizando traslaciones, rotaciones y oclusiones parciales. Este úlimo modelo obtuvo una mejora considerable de los resultados con respecto a los anteriores y fue el elegido como modelo final.
\end{enumerate}

\medskip

\noindent En general, los resultados obtenidos son prometedores, ya que se obtiene una gran tasa de acierto en el problema de regresión. Como se mencionó al comienzo del trabajo este enfoque puede ser de gran utilidad a expertos en Antropología Forense para el marcado de landmarks cefalométricos en el proceso de la \textit{Superposición Craneofacial}, sin embargo, remarcamos la importancia de una supervisión del trabajo de marcado por un experto, ya que no se han tenido en cuenta los márgenes de error admisibles en cada landmark.

\section{Trabajos Futuros}

\noindent Ante los resultados obtenidos en esta investigación, posibles vías de trabajo futuras podrían ser:

\begin{enumerate}
    \item La integración del modelo en un proceso real de detección del landmarks cefalométricos por parte de un equipo de expertos y estudiar su rendimiento y precisión.
    \item Realizar un ajuste fino previo de la parte no supervisada de la red para aprender a mejorar la reconstrucción de caras en imágenes de baja calidad, en escalas de grises y en distintas posiciones y ver el impacto de la mejora de la reconstrucción de las imágenes en el marcado de landmarks.
    \item Extender el conjunto de datos proporcionado con el fin de mejorar aún más la precisión del marcado con más ejemplos de entrenamiento para el modelo.
\end{enumerate}


\section{Sobre el TFG}

\noindent 


\endinput
%------------------------------------------------------------------------------------
% FIN DEL CAPÍTULO. 
%------------------------------------------------------------------------------------

