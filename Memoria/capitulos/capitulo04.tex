

\chapter{Conclusiones y Trabajos futuros}

\noindent En el primer capítulo dimos una introducción a las \textbf{CNN} y establecimos dos objetivos principales:

\begin{itemize}
  \item Tratar de dar una posible modelización matemática para una CNN.
  \item Demostrar la propiedad básica de la invarianza por traslaciones. 
\end{itemize}


\medskip

\noindent En primer lugar expusimos el problema de conseguir un operador adecuado que se pudiera aplicar de manera recursiva del mismo modo que hacen las CNN. Para ello se llegó a la conclusión de que dicho operador debía ser \textbf{invariante por traslaciones} y \textbf{Lipschitz-continuo} bajo la acción de difeomorfismos y vimos que debíamos rechazar el operador módulo de la transfomada de Fourier como candidato por no cumplir esta segunda propiedad. 

\medskip

\noindent Una vez rechazado el operador anterior vimos que la alternativa más prometedora era la de usar una transformada de Ondeletas, en concreto la de \textbf{Littlewood-Paley}:

\begin{equation}
  \forall x \in  \mathbb{R}^d \;\; W[\lambda]f(x)= f \ast \psi_\lambda(x)=\int f(u)\psi_\lambda(x-u) du .
\end{equation}

\noindent Este operador, al contrario que el módulo de la transformada de Fourier, es Lipschitz-continuo bajo la acción de difeomorfismos, pero produce coeficientes que no son invariantes por traslaciones. Para ello se necesita la ayuda de un operador no lineal auxiliar $M[\lambda]W[\lambda]f=|W[\lambda]f|$ y vimos que usando el módulo conseguíamos obtener un operador invariante por traslaciones definido en cualquier camino $p \in \mathcal{P}_\infty$ como:

\begin{equation}
  \overline{S}f(p)=\int_{\mathbb{R}^d}U[p]f(x)dx 
\end{equation} 

\medskip

\noindent No obstante, en la práctica se empleará el operador de ventana:

\begin{equation}
  S_J[p]f(x)=\left| |f \ast \psi_{\lambda_1} | \ast \psi_{\lambda_2} | \ldots | \ast \psi_{\lambda_m} \right| \ast \phi_{2^J}(x).
\end{equation}

\noindent Con este operador se consigue definir el que será la modelización de una CNN, y que además tiene unas propiedades deseables como la preservación de la norma o que es un operador no expansivo. Finalmente, con la modelización propuesta (el operador de ventana) se comprueba que no es expansivo en conjuntos de caminos a medida que decrece el umbral y con esta propiedad se consigue demostrar la invarianza por traslaciones del operador.

\medskip

\noindent Llegados a este punto podemos afirmar haber conseguido todos los objetivos que nos habíamos planteado. Sin embargo el camino para lograrlos ha sido complicado y ha estado marcado por la consecución de retos menores que me han hecho aprender mucho sobre conceptos que se explicaron durante el grado y otros totalmente nuevos.

\medskip

\noindent En primer lugar tuve que recordar conceptos básicos del análisis de Fourier como son el cálculo de series de Fourirer y transformada de Fourier así como sus interpretaciones y aplicaciones. Por otro lado tuve que recordar conceptos de análisis como son la Lipschitz-continuidad y con esto pude entender y explicar la demostración de que el módulo de la transformada de Fourier no es lipschitz-continuo bajo la acción de difeomorfismos. También he aprendido teoremas conocidos en el ámbito del análisis de Fourier como son la fórmula de Plancharel o el Teorema de convolución de la transformada de Fourier. 

\medskip

\noindent Todas estas herramientas me han permitido introducirme en el mundo de las ondeletas, un mundo desconocido para mi y que es esencial en áreas de conocimiento como el procesamiento de señales y la visión por computador. Tuve que aprender y entender el concepto de ondeleta y transformada de ondeletas, y tuve que realizar una investigación sobre el tratamiento de señales (especialmente de imágenes) por medio de estas heramientas.

\medskip

\noindent Por otro lado, esta investigación me ha ayudado a comprender mejor el funcionamiento de una red neuronal convolucional y las posibles motivaciones matemáticas y propiedades subyacentes. Este tipo de estudio, de nuevo da sentido y sirve como nexo de unión a los dos grados que he cursado. Me he habituado a consultar y leer publicaciones matemática que trataran sobre estos temas, una destreza que considero de vital importancia y que sin duda será de gran utilidad para el futuro.

\endinput