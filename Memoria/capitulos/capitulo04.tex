

\chapter{Conclusiones y Trabajos futuros}

\noindent En el primer capítulo dimos una introducción a las \textbf{CNN} y establecimos dos objetivos principales:

\begin{itemize}
  \item Tratar de dar una posible modelización matemática para una CNN.
  \item Demostrar la propiedad básica de la invarianza por traslaciones. 
\end{itemize}

\noindent Llegados a este punto podemos afirmar haber conseguido todos los objetivos que nos habíamos marcado, sin embargo el camino para lograrlos ha sido complicado y ha estado marcado por la consecución de retos menores que me han hecho aprender mucho sobre conceptos que se explicaron durante el grado y que a continuación detallaré. 

\medskip

\noindent En primer lugar tuve que recordar conceptos básicos del análisis de Fourier como son el cálculo de series de Fourirer y transformada de Fourier así como sus interpretaciones, por otro lado tuve que recordar conceptos de análisis como son la Lipschitz-continuidad y con esto pude entender y explicar mejor la demostración de que el módulo de la transformada de Fourier no es lipschitz-continuo bajo la acción de difeomorfismos.

\medskip

\noindent En segundo lugar, tuve que aprender un concepto nuevo para mi como es el de ondeleta y transformada de ondeletas, y tuve que realizar una investigación sobre el tratamiento de señales (especialmente de imágenes) por medio de estas heramientas. Estos conceptos sientan las bases de la \textbf{Visión por Computador}, rama en la que las CNN han adquirido una importancia muy notable.

\medskip

\noindent En tercer lugar tuve que habituarme a consultar y leer publicaciones matemática que trataran sobre estos temas, una destreza que considero de vital importancia y que ha provocado que en las búsquedas que realizo en internet prefiera consultar artículos científicos antes que otro tipo de entradas.

\medskip

\noindent En cuarto lugar, es la prmiera vez que me enfrento a un trabajo de iniciación a la investigación, y por lo tanto en muchas situaciones me he visto envuelto de conceptos que no comprendía del todo y de los cuales no había mucha información. También he tenido que clasificar la información siempre pensando en el trabajo final y en los objetivos marcados. 

\medskip

\noindent Todos estos objetivos, como he mencionado antes, han sido necesarios para poder terminar el trabajo con éxito por lo que quería hacer mención a ellos.

\subsection*{Trabajos Futuros}

\noindent Si bien hemos conseguido proporcionar una posible modelización para las CNN y demostrar la invarianza por traslaciones, quedan aún propiedades importantes por comprobar como son la \textbf{invarianza frente a pequeños difeomorfismos}, propiedad que presentan las CNN tal y como las conocemos. Por otro lado se podría tratar de investigar sobre el teorema de aproximación universal, etc...

\endinput