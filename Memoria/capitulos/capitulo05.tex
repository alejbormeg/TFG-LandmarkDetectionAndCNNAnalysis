% !TeX root = ../libro.tex
% !TeX encoding = utf8

\chapter{Introducción}

% Así se crea una cita\dictum[Leonard Euler]{Mathematicians have tried in vain to this day to discover some order in the sequence of prime numbers, and we have reasons to believe that it is a mystery into which the human mind will never penetrate.}

\section{Introducción}

Las \textbf{ciencias forenses} son aquellas que aplican el método científico a hechos presuntamente delictivos con la finalidad de aportar pruebas a efectos judiciales. Este campo es interdisciplinar que incluye principalmente a la Criminalística\footnote{Disciplina encargada del descubrimiento y verificación científica de presuntos hechos delictivos y quienes los cometen.} y la Medicina Forense\footnote{Disciplina encargada de determinar el origen de las lesiones, las caisas de muerte o la identificación de seres humanos vivos o muertos.}. 

\medskip

\noindent Así pues, este trabajo ubica en el ámbito de la \textbf{antropología forense}, que es una rama de la Medicina Forense que se encarga de determinar la edad, raza, sexo o estatura, entre otras, a partir de restos óseos en problemas de reconstrucción facial, identificación de víctimas en desastres en masa o en identificación facial. 


\subsection{Descripción del problema}

La \textbf{Superposición Craneofacial} es una técnica de identificación forense mediante la cual se comparan imágenes de la persona difunta\footnote{A estas imágenes se le denominan imágenes ante-mortem} con una o varias imágenes de un cráneo candidato. La técnica empleada es la superposición de ambas imágenes y se estima si son o no la misma persona de acuerdo a correspondencias morfológicas o marcando puntos de referencia. Los \textit{landmarks} o puntos de referencia, pueden situarse en el cráneo\footnote{En este caso reciben el nombre de puntos craneométricos} encontrado o en el rostro\footnote{En este caso se denominan puntos cefalométricos}. Entre los dos tipos de \textit{landmarks} anteriores existe una correlación, en caso de pertenecer a la misma persona, que el antropólogo forense trata de descubrir. 

\noindent Esta tarea no es sencilla debido al \textbf{tejido blando facial} que separa el punto craneométrico de su homólogo cefalométrico y que lo desplaza. El desplazamiento ocasionado por el tejido blando facial no es constante ni se produce siempre en la misma dirección, lo cual junto con otros factores como la grasa o la calidad de la imagen complica esta tarea de superponer las dos imágenes (de cráneo y cara) con fidelidad.

\medskip

\noindent Tradicionalmente, el proceso era esencialmente manual y complicado de replicar, y pese a los avances actuales que se están llevando a cabo para automatizar esta tarea \cite{Huete2015PastPA}, la identificación de \textit{landmarks} sigue realizándose a mano normalmente. 

\medskip

\noindent En este contexto, el presente trabajo se centrará en esta etapa del marcado de \textit{landmarks}, en concreto de \textbf{landmarks cefalométricos} (en las imágenes ante-mortem). El objetivo será comparar dos frameworks que utilizan técnicas de \textbf{Deep Learning} para la detección y marcado de \textbf{landmarks cefalométricos}  
\subsection{Motivación}
\subsection{Objetivos}


%\dictum[Leonard Euler]{The evaluation of the correspondence between the face and the skull is the result of the comparison between anatomical landmarks, morphological features and anthropometric measurements. In photographic CFS techniques, this crucial stage relies on the visual observation of the expert without any technological support.}


\endinput
%------------------------------------------------------------------------------------
% FIN DEL CAPÍTULO. 
%------------------------------------------------------------------------------------


