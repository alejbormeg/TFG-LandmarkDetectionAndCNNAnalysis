

\chapter{Invarianza por Traslaciones} \label{ch:seccion14}

\noindent Hasta ahora hemos definido el propagador de dispersión y hemos visto algunas propiedades como la conservación de la norma de la señal $f$. No obstante, aún queda por demostrar la invarianza por traslaciones.

\section{No expansividad del operador de ventana en conjuntos de caminos}
Vamos a demostrar en primer lugar que $||S_J[\mathcal{P}_J] f- S_J[\mathcal{P}_J] h ||$ es no expansiva cuando se incrementa $J$, y que de hecho converge cuando $J \rightarrow \infty$. Esto define una distancia límite que como veremos a continuación es invariante por traslaciones.

\medskip

\begin{proposicion}
\noindent Para $f,h \in L^2(\mathbb{R}^d)$ y $J\in \mathbb{Z}$ se cumple

\begin{equation} \label{eq::1.10}
  || S_{J+1} [\mathcal{P}_{J+1}]f- S_{J+1}[\mathcal{P}_{J+1}]h || \leq ||S_J[\mathcal{P}_J]f - S_J[\mathcal{P}_J]h || 
\end{equation}

\end{proposicion}

\begin{proof}

\noindent En primer lugar, vamos a transformar la condición que queremos demostrar en \eqref{eq::1.10} en otra equivalente y que será más fácil de probar.

\medskip

\noindent Si recordamos la definición de $\mathcal{P}_J$, era un conjunto de caminos finitos $p=(\lambda_1,\ldots,\lambda_m)$ tal que $\lambda_k\in\Lambda_J$ y $|\lambda_k|=2^{k}>2^{-J}$. Luego todo camino $p' \in \mathcal{P}_{J+1}$, puede ser unívocamente escrito como una extensión de un camino $p\in \mathcal{P}_J$ donde $p$ es el prefijo más grande de $p'$ que pertenece a $\mathcal{P}_J$, y $p'=p+q$ para algún $q\in \mathcal{P}_{J+1}$. De hecho, podemos definir el conjunto de todas las extensiones de $p\in \mathcal{P}_J$ en $\mathcal{P}_{J+1}$ como: 

\begin{equation}
  \mathcal{P}_{J+1}^{p}={p} \cup {p+2^{-J}r+p''} \; \; r\in G^{+},p''\in \mathcal{P}_{J+1}.
\end{equation}

\noindent Esto define una partición disjunta de $\mathcal{P}_{J+1}=\cup_{p \in \mathcal{P}_J} \mathcal{P}_{J+1}^{p}$. Y deberíamos probar que dichas extensiones son no expansivas,

\begin{equation}\label{eq::1.11}
  \sum_{p' \in \mathcal{P}_{J+1}^p} || S_{J+1}[p']f-S_{J+1}[p']h||^2 \leq ||S_{J}[p]f-S_J [p]h||^2.
\end{equation}

\noindent Finalmente, si nos fijamos, la condición \eqref{eq::1.11} equivale a \eqref{eq::1.10} sumando en todo $p\in \mathcal{P}_J$, luego probando \eqref{eq::1.11} tendríamos el resultado que buscamos. 


\medskip


\noindent Para ello vamos a necesitar el siguiente lema:

\begin{lema}
  Para ondeletas que satisfacen la propiedad presentada en la Proposición \ref{unitario}, para toda función \textbf{real} $f\in L^2(\mathbb{R}^d)$ y todo $q \in \mathbb{Z}$ se verifica: 

  $$\sum_{-q\geq l > -J} \sum_{r \in G^+} || f \ast \psi_{2^lr}||^2 + || f \ast \phi_{2^J}||^2 = || f \ast \phi_{2^q} ||^2$$

\end{lema}

\begin{proof}

  \noindent En primer lugar vamos a ver que de la Proposición \ref{unitario} se deduce la siguiente expresión: 

  $$|\widehat{\phi}(2^J \omega)|^2 + \sum_{-q\geq l > -J} \sum_{r \in G^+}|\widehat{\psi}(2^{-l}r^{-1} \omega)|^2=|\widehat{\phi}(2^q \omega) |^2.$$

  \noindent Para ello, de la expresión

  \begin{align*}
    \frac{1}{2} \sum_{j=-\infty}^\infty \sum_{r \in G} |\widehat{\psi}(2^{-j}r^{-1}\omega)|^2=1 & \; \; y
    \;\;|\widehat{\phi}(\omega)|^2= \frac{1}{2} \sum_{j=-\infty}^0 \sum_{r\in G} |\widehat{\psi}(2^{-j}r^{-1}\omega)|^2,
  \end{align*}

  \noindent se tiene de la misma forma que vimos en la demostración de la Proposición \ref{unitario} que:

  $$\forall J \in \mathbb{Z} \; \; \; \left|\widehat{\phi}\left(2^J\omega\right)\right|^2 + \frac{1}{2} \sum_{j>-J,r\in G}\left|\widehat{\psi}\left(2^{-j}r^{-1}\omega\right)\right|^2=1. $$

  \noindent Y partiendo el sumatorio obtenemos que: 

  $$\left|\widehat{\phi}\left(2^J\omega\right)\right|^2 + \frac{1}{2}  \sum_{-q \geq j >-J,r \in G}\left|\widehat{\psi}\left(2^{-j}r^{-1}\omega\right)\right|^2= \frac{1}{2} \sum_{j>-q,r \in G}\left|\widehat{\psi}\left(2^{-j}r^{-1}\omega\right)\right|^2=|\widehat{\phi}(2^q \omega)|^2.$$
  
  \noindent Ahora multiplicamos la expresión anterior por $|\widehat{f}(\omega)|^2$, 
  
  $$\left|\widehat{f}(\omega)\right|^2 \left|\widehat{\phi}\left(2^J\omega\right)\right|^2 + \frac{1}{2} \sum_{-q \geq j >-J,r \in G} \left|\widehat{f}(\omega)\right|^2 \left|\widehat{\psi}\left(2^{-j}r^{-1}\omega\right)\right|^2=\left|\widehat{f}(\omega)\right|^2 \left|\widehat{\phi}(2^q \omega)\right|^2.$$

  \noindent Integramos en $\omega$, 

  \begin{align*}
    \int \left|\widehat{f}(\omega)\right|^2 \left|\widehat{\phi}\left(2^J\omega\right)\right|^2 d\omega &+ \frac{1}{2} \sum_{-q \geq j >-J,r \in G} \int \left|\widehat{f}(\omega)\right|^2 \left|\widehat{\psi}\left(2^{-j}r^{-1}\omega\right)\right|^2 d\omega \\ 
    &=\int \left|\widehat{f}(\omega)\right|^2 \left|\widehat{\phi}(2^q \omega)\right|^2 d\omega.
  \end{align*}


  \noindent Ahora estamos en condiciones de aplicar el \autoref{Teorema::Convolucion}, y nos queda que la expresión anterior equivale a: 

  \begin{align*}
    \int \left|(f \ast \phi_{2^J}) (x)\right|^2 dx + \sum_{-q \geq j >-J,r \in G} \int \left|(f \ast \psi_{2^{j}r})(x)\right|^2 dx=\int \left|(f \ast \phi_{2^q}) (x)\right|^2 dx,
  \end{align*}

  \noindent y teniendo en cuenta que $f$ es real y por lo tanto que $||f \ast \psi_{2^j r}||= ||f \ast \psi_{2^j -r}||$ junto con la defnición de la norma de $L^2(\mathbb{R}^d)$, se tiene  

  \begin{align*}
    \sum_{-q\geq l > -J} \sum_{r \in G^+} || f \ast \psi_{2^lr}||^2 + || f \ast \phi_{2^J}||^2 = || f \ast \phi_{2^q} ||^2& \\ 
    &\qedhere
  \end{align*} 

\end{proof}


\noindent Vamos ahora a usar el lema anterior con la función $g=U[p]f-U[p]h$ junto con que $U[p]f\ast \phi_{2^J}=S_J[p]f$. De esta forma se tiene:

$$||g \ast \phi_{2^{J+1}} ||^2 + \sum_{r\in G^+} || g\ast \psi_{2^{-J}r} ||^2=||g \ast \phi_{2^J}||^2.$$

\noindent Así, sustituyendo el valor de $g$ por el que hemos definido antes y aplicando la propiedad distributiva de la convolución, obtenemos:

\begin{align*}
  ||U[p]f \ast \phi_{2^J} -U[p]h \ast \phi_{2^J}||^2 = &||U[p]f \ast \phi_{2^{J+1}} - U[p]h \ast \phi_{2^{J+1}} ||^2 \\
  & + \sum_{r\in G^+} || U[p]f\ast \psi_{2^{-J}r} -U[p]h \ast \psi_{2^{-J}r}||^2.
\end{align*}

\noindent Y esto equivale a

\begin{align*}
  ||S_{J}[p]f-S_J[p]h||^2 = &|| S_{J+1}[p]f-S_{J+1}[p]h||^2 \\
  & + \sum_{r\in G^+} || U[p]f\ast \psi_{2^{-J}r} -U[p]h \ast \psi_{2^{-J}r}||^2.
\end{align*}

\noindent Aplicando ahora la propiedad de la norma de que $\|g - h \| \geq \| |g| - |h| \|$ y como $$|U[p]f\ast\psi_{2^{-J}r}|=U[p+2^{-J}r]f,$$ se concluye que

\begin{align*}
    ||S_{J}[p]f-S_J[p]h||^2 \geq & || S_{J+1}[p]f-S_{J+1}[p]h||^2 \\
    & + \sum_{r\in G^+} ||U[p+2^{-J}r]f-U[p+2^{-J}r]h||^2.
\end{align*}


\noindent Como $S_{J+1}[\mathcal{P}_{J+1}]U[p+2^{-J}r]f=\lbrace S_{J+1} [p+2^{-J}r+p'']\rbrace_{p''\in\mathcal{P}_{J+1}}$  y $S_{J+1}[\mathcal{P}_{J+1}]f$ es no expansiva por la Proposición \ref{proposicion::NoExpansiva}, usando la desigualdad anterior podemos escribir

\begin{align*}
    ||S_{J}[p]f - S_J[p]h||^2 \geq & ||S_{J+1}[p]f-S_{J+1}[p]h||^2 \\
    & + \sum_{p''\in \mathcal{P}_{J+1}} \sum_{r\in G^+} || S_{J+1}[p+2^{-J}r+p'']f- S_{J+1}[p+2^{-J}r+p'']h||^2,
\end{align*}

\noindent y en particular

\begin{align*}
  ||S_{J}[p]f - S_J[p]h||^2 \geq & \sum_{p''\in \mathcal{P}_{J+1}} \sum_{r\in G^+} || S_{J+1}[p+2^{-J}r+p'']f- S_{J+1}[p+2^{-J}r+p'']h||^2,
\end{align*}

\noindent que demuestra \eqref{eq::1.11}. \qedhere
\end{proof}


\section{Invarianza por traslaciones}

\noindent La proposición anterior nos demuestra que $||S_J[\mathcal{P}_J]-S_J[\mathcal{P}_J]h||$ es positivo y no creciente cuando $J$ se incrementa, y de hecho converge. Como $S_J[\mathcal{P}_J]$ es no expansiva, el límite también lo es: 

$$\forall f,h \in L^2(\mathbb{R}^d) \lim_{J\rightarrow\infty} ||S_J[\mathcal{P}_J]f-S_{J}[\mathcal{P}_J]h|| \leq ||f-h||.$$

\medskip

\noindent Para ondeletas de dispersión admisibles que satisfacen \eqref{eq::1.7}, el \autoref{teoremaOndeletasAdmisibles} nos demuestra que si $||S_J[\mathcal{P}_J]f||=||f||$ entonces $\lim_{J\rightarrow\infty}||S_J[\mathcal{P}_J]f||=||f||$. El siguiente teorema demuestra que el límite es invariante por traslaciones, pero para la demostración del teorema necesitaremos dos resultados auxiliares:

\begin{lema} \label{lemma:Schur}
  Para cualquier operador $Kf(x)=\int f(u)k(x,u)du \; \in L^2(\mathbb{R}^d)$ se tiene 

  $$\int |k(x,u)| dx \leq C,$$

  \noindent con $C >0$, y además 

  $$\int |k(x,u)|du \leq C \implies ||K||\leq C.$$

  \noindent Donde $||K||$ es la norma en $L^2(\mathbb{R}^d)$ de $K$.
\end{lema}

\medskip

\noindent Para la demostración del lema anterior se necesita el Lema de Schur \cite{SchurLemma}. Debido a que se trata de un lema auxiliar que emplearemos para la demostración del teorema principal de la sección, no lo demostraremos.
 
\begin{lema} \label{lema::constante}
  Existe una constante $C$ tal que para todo $\tau \in \C^2(\mathbb{R}^d)$ con $||\nabla \tau ||_\infty \leq \frac{1}{2}$ se tiene que 
  
  $$||L_\tau A_J f - A_J f|| \leq C ||f||2^{-J}||\tau||_{\infty}.$$
\end{lema}

\begin{proof}

  \noindent En esta prueba, vamos a necesitar el Lema\ref{lemma:Schur}. 

  \medskip

  \noindent El operador norma de $k_J=L_\tau A_J - A_J$ se calcula aplicando el lema de Schur a su kernel,

  $$k_J(x,u)=\phi_{s^J}(x-\tau(x)-u)-\phi_{2^J}(x-u).$$

  \noindent Si nos fijamos en la expresión anterior, cuando $x=0=u$ se tiene que: 

  $$k_J(0,0)=\phi_{2^J}(0)-\phi_{2^J}(0)=0.$$

  \noindent Si ahora calculamos su polinomio de Taylor de primer orden centrado en el $(0,0)$ se obtiene: 

  $$k_J=k_J(0,0)+\int_0^1 \nabla \phi_{2^J} (x-t\tau(x)-u)\tau(x). dt$$

  \noindent Si ahora calculamos el módulo obtenemos que:

  \begin{align*}
    |k_J|&=|k_J(0,0)+\int_0^1 \nabla \phi_{2^J} (x-t\tau(x)-u)\tau(x) dt| \\
    &\leq |k_J(0,0)|+\left|\int_0^1 \nabla \phi_{2^J} (x-t\tau(x)-u)\tau(x) dt \right| \\
    & \leq \left|\int_0^1 \nabla \phi_{2^J} (x-t\tau(x)-u)\tau(x) dt \right| \\ 
    & \leq \int_0^1 \left| \nabla \phi_{2^J} (x-t\tau(x)-u)\tau(x)  \right|dt= |\tau(x)|\int_0^1 \left| \nabla \phi_{2^J} (x-t\tau(x)-u)\right|dt  \\ 
    & \leq ||\tau(x)||_\infty \int_0^1 \left| \nabla \phi_{2^J} (x-t\tau(x)-u)\right|dt.
  \end{align*}

  \noindent Si ahora integramos en $u$ y aplicamos el teorema de Fubini para itercambiar las integrales del lado derecho de la desigualdad obtenemos:

  \begin{align*}
    \int |k_J| du &\leq ||\tau(x)||_\infty \int \int_0^1 \left| \nabla \phi_{2^J} (x-t\tau(x)-u)\right|dt \; du = ||\tau(x)||_\infty \int_0^1 \int  \left| \nabla \phi_{2^J} (x-t\tau(x)-u)\right| du \; dt. 
  \end{align*}

  \noindent por otro lado, vamos a comprobar que 
  $$\nabla\phi_{2^J}(x)=2^{-dJ-J} \nabla \phi(2^{-J}x).$$

  \noindent Para ello debemos recordar que $\phi_{2^J}(x)=2^{-dJ}\phi(2^{-J}x)$ luego

  \begin{align*}
    \nabla \phi_{2^J}(x) &= \nabla(2^{-dJ}\phi(2^{-J}x)) \\
    &= 2^{-dJ} \nabla(\phi(2^{-J}x)).
  \end{align*}

  \noindent Si nos fijamos, debido a que $x$ está multiplicado por $2^{-J}$ en cada componente del vector, siempre que derivemos con respecto a alguna componente, vamos a poder sacar como factor común $2^{-J}$ por lo tanto:

  \begin{align*}
    \nabla \phi_{2^J}(x) = 2^{-dJ-J} \nabla(\phi(2^{-J}x)).
  \end{align*}

  \noindent De esta forma, realizando un cambio de variable obtenemos: 

  \begin{align*}
    \int |k_J| du &\leq ||\tau(x)||_\infty  2^{-dJ-J} \int \left| \nabla \phi (2^J u')\right|du' \\
    & = 2^{-J} ||\tau(x)||_\infty ||\nabla\phi||_1. 
  \end{align*}

  \noindent Si ahora realizamos el mismo procedimiento integrando en $x$ en vez de en $u$ tenemos que 

  \begin{align*}
    \int |k_J(x,u)| dx &\leq ||\tau(x)||_\infty \int_0^1 \int \left| \nabla \phi_{2^J} (x-t\tau(x)-u)\right| dx \; dt
  \end{align*}

  \noindent Ahora, aplicamos el cambio de variable $v=x - t\tau(x)$ y calculamos su Jacobiano

  \begin{align*}
    Jv& =J(x-t\tau(x))=J(x)-J(t\tau(x)) \\
    & = Id - tJ(\tau(x)) \\
    & = Id - t\nabla \tau(x).
  \end{align*}

  \noindent Vamos a buscar una cota para el determinante del Jacobiano

  \begin{align*}
    |J| & =(1-t\tau(x))^d \\
    & \geq (1-||\tau||_\infty)^d \\
    & \geq 2^{-d}.
  \end{align*}

  \noindent Aplicando ahora el cambio de variable a la integral

  \begin{align*}
    \int |k_J(x,u)| dx &\leq ||\tau(x)||_\infty 2^d \int_0^1 \int \left| \nabla \phi_{2^J} (v-u)\right| dv \; dt \\
    &= 2^{-J} ||\tau||_\infty ||\nabla \phi ||_1 2^{d}.
  \end{align*}

  \noindent De las dos cotas superiores obtenidas esta es la mayor, por lo que aplicamos el lema de Schur a esta y terminamos la demostración del lema

  $$ ||L_{\tau} A_J - A_J || \leq 2^{-J+d} ||\nabla \phi ||_1 ||\tau||_\infty. \;\;\; \qedhere$$
\end{proof}


\noindent Con esto ya tenemos todas las herramientas necesarias para enunciar y demostrar el teorema central de esta sección, que nos garantiza que el operador que estamos construyendo y que modeliza una red neuronal convolucional, es invariante por traslaciones.

\begin{teorema} \label{invarianzaTraslaciones}
Para ondeletas de dispersión admisibles se tiene que 

$$\forall f \in L^2(\mathbb{R}^d), \; \forall c\in \mathbb{R}^d \;\;\; \lim_{J\rightarrow \infty}||S_J[\mathcal{P}_J] f-S_J[\mathcal{P}_J] L_cf||=0$$
\end{teorema}

\begin{proof}

\noindent Fijamos $f\in L^2(\mathbb{R}^d)$. Teniendo en cuenta la conmutatividad $S_J[\mathcal{P}_J] L_cf = L_cf S_J[\mathcal{P}_J]$ y la definición $S_J[\mathcal{P}_J]f=A_J U[\mathcal{P}_J]f$, obtenemos

\begin{align*}
    ||S_J[\mathcal{P}_J] L_cf - S_J[\mathcal{P}_J]f || &= ||L_c A_J U[\mathcal{P}_J]f - A_J U[\mathcal{P}_J]f|| \\
    &\leq ||L_c A_J - A_J|| ||U[\mathcal{P}_J]f||.
\end{align*}

\medskip

\noindent Si ahora aplicamos el Lema\ref{lema::constante} con $\tau=c$, se tiene que $||\tau||_\infty=|c|$ y además

$$||L_c A_J - A_J|| \leq C 2^{-J} |c|.$$

\noindent Y si tenemos en cuenta esto en la expresión anterior nos da que: 

\begin{align*}
  ||S_J[\mathcal{P}_J] L_cf - S_J[\mathcal{P}_J]f || & \leq ||L_c A_J - A_J|| ||U[\mathcal{P}_J]f|| \\
  & \leq C 2^{-J} |c| ||U[\mathcal{P}_J]f||
\end{align*}

\noindent Como la admisibilidad de la condición \eqref{eq::1.7} se satisface, en el Lema \ref{lema::Admisibilidad} se demuestra en \eqref{eq::1.9} que para $J>1$ se cumple

$$\frac{\alpha}{2}||U[\mathcal{P}_J]f||^2 \leq (J+1)||f||^2+||f||^2_w.$$

\noindent Y de esta expresión podemos sacar una cota superior para $||U[\mathcal{P}_J]f||$: 

\begin{align*}
  ||U[\mathcal{P}_J]f||^2 \leq ((J+1)||f||^2+||f||^2_w) 2 \alpha^{-1}
\end{align*}

\noindent Si $||f||_w < \infty$ entonces elevando al cuadrado en la desigualdad de antes tenemos

$$||S_J[\mathcal{P}_J] L_cf - S_J[\mathcal{P}_J]f ||^2 \leq ((J+1)||f||^2+||f||_w^2)C^2 2 \alpha^{-1} 2^{-2J} |c|^2,$$

\noindent y tomando límite en ambos lados cuando $J\rightarrow \infty$ tenemos que 

\begin{align*}
  \lim_{J\rightarrow \infty} ||S_J[\mathcal{P}_J] L_cf - S_J[\mathcal{P}_J]f ||^2 &\leq \lim_{J\rightarrow \infty} ((J+1)||f||^2+||f||_w^2)C^2 2 \alpha^{-1} 2^{-2J} |c|^2 \\
  &= 0.
\end{align*}

\noindent Luego $\lim_{J\rightarrow\infty}||S_J[\mathcal{P}_J] L_cf - S_J[\mathcal{P}_J]f ||=0$.

\medskip

\noindent Finalmente vamos a probar ahora que el límite anterior se da $\forall f \in L^2(\mathbb{R}^d)$, con un argumento similar al de la prueba del \autoref{teoremaOndeletasAdmisibles}. Cualquier $f\in L^2(\mathbb{R}^d)$ se puede escribir como el límite de una sucesión de funciones $\lbrace f_n \rbrace_{n\in\mathbb{N}}$ con $||f_n||_w < \infty$, y como $S_J[\mathcal{P}_J]$ es no expansivo y $L_c$ es unitario, usando la desigualdad triangular, deducimos que 

$$||L_c S_J[\mathcal{P}_J]f-S_J[\mathcal{P}_J]f|| \leq ||L_c S_J [\mathcal{P}_J]f_n -S_J[\mathcal{P}_J]f_n|| + 2||f-f_n||.$$

\noindent Haciendo tender $n \rightarrow \infty$ se prueba que $\lim_{J\rightarrow \infty}||S_J[\mathcal{P}_J] f-S_J[\mathcal{P}_J] L_cf||=0$ con lo que acaba la demostración. \qedhere
\end{proof}


\endinput