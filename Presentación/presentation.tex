% !TeX document-id = {2870843d-1baa-4f6a-bd0a-a5c796104a32}
% !BIB TS-program = biber
% !TeX encoding = UTF-8
% TU Delft beamer template

\documentclass[aspectratio=43]{beamer}
\usepackage[english]{babel}
\usepackage{csquotes}
\usepackage{calc}
\usepackage[absolute,overlay]{textpos}
\usepackage{graphicx}
\usepackage{subfig}
\usepackage{mathtools}
\usepackage{amsfonts}
\usepackage{amsthm}
\usepackage{comment}
\usepackage{siunitx}
\usepackage{MnSymbol,wasysym}
\usepackage{array}
\usepackage{qrcode}
\useoutertheme[subsection=false]{miniframes}

\setbeamertemplate{navigation symbols}{} % remove navigation symbols
\mode<presentation>{\usetheme[verticalbar=false]{tud}}

% BIB SETTINGS
\usepackage[
    backend=biber,
    giveninits=true,
    maxnames=30,
    maxcitenames=20,
    uniquename=init,
    url=false,
    style=authoryear,
]{biblatex}
\addbibresource{bibfile.bib}
\setlength\bibitemsep{0.3cm} % space between entries in the reference list
\renewcommand{\bibfont}{\normalfont\scriptsize}
\setbeamerfont{footnote}{size=\tiny}
\renewcommand{\cite}[1]{\footnote<.->[frame]{\fullcite{#1}}}
\setlength{\TPHorizModule}{\paperwidth}
\setlength{\TPVertModule}{\paperheight}

\newcommand{\absimage}[4][0.5,0.5]{%
	\begin{textblock}{#3}%width
		[#1]% alignment anchor within image (centered by default)
		(#2)% position on the page (origin is top left)
		\includegraphics[width=#3\paperwidth]{#4}%
\end{textblock}}

\newcommand{\mininomen}[2][1]{{\let\thefootnote\relax%
	\footnotetext{\begin{tabular}{*{#1}{@{\!}>{\centering\arraybackslash}p{1em}@{\;}p{\textwidth/#1-2em}}}%
	#2\end{tabular}}}}

\title[]{Localización de landmarks cefalométricos por medio de técnicas de few-shot learning y análisis de redes convolucionales}
\institute[]{\textbf{Tutores}: Pablo Mesejo Santiago, Javier Merí de la Maza \and Universidad de Granada, España}
\author{Alejandro Borrego Megías}
\date{\today}

\begin{document}

{
\setbeamertemplate{footline}{\usebeamertemplate*{minimal footline}}
\frame{\titlepage}
}

% Parte de Matemáticas
\part{Análisis de Redes convolucionales}

\begin{frame}{Índice Primera Parte}
  \textcolor{tudCyan}{\textbf{Análisis de redes convolucionales}}
  \medskip
  \tableofcontents[part=1]
\end{frame}

% Current section
\AtBeginSection[ ]
{
\begin{frame}{Índice Primera Parte}
    \tableofcontents[currentsection]
\end{frame}
}

\section{Introducción}

\section{Modelización}

\section{Invarianza por traslaciones}

\section{Conclusiones}

% Parte de Informática
\part{Localización de landmarks cefalométricos por medio de técnicas de few-shot learning}

\begin{frame}{Índice Segunda Parte}
  \textcolor{tudCyan}{\textbf{Localización de landmarks cefalométricos por medio de técnicas de few-shot learning}}
  \medskip
  \tableofcontents[part=2]
\end{frame}

% Current section
\AtBeginSection[ ]
{
\begin{frame}{Índice Segunda Parte}
    \tableofcontents[currentsection]
\end{frame}
}

\section{Introducción}

\section{Fundamentos Teóricos}

\section{Estado del arte}

\section{Experimentos}

\section{Conclusiones}


\begin{frame}[fragile]{Example frame 1} % some commands, e.g. \verb require [fragile]
This is the first frame.

You can set the blue bar vertical using the option \verb|\usetheme[verticalbar=true]{tud}|.

Set the aspect ratio to 4:3 with the
documentclass option aspectratio=43. Use aspectratio=169 for wide screen (16:9).
\end{frame}

\section{Examples}
\begin{frame}{Example frame 2}
  \begin{block}{Block}
    \begin{itemize}
      \item item 1
      \item item 2
    \end{itemize}
  \end{block}

  \begin{exampleblock}{Example}
    \begin{enumerate}
      \item Sugar in a stirred cup of tea gathers in the middle.
      \item Rivers often take a detour through flat terrain.
    \end{enumerate}
  \end{exampleblock}

  \begin{alertblock}{Alert}
     Rivers and sweet tea do unexpected things.\cite{Einstein1926}
  \end{alertblock}
\end{frame}

% \begin{frame}{Mass--energy equivalence}
% 	They say every formula you add to a presentation, will reduce your audience by \SI{50}{\percent}. A simple yet effective way to mitigate this effect, is adding a compact nomenclature to the slides containing formulae.
	
% 	\[E=mc^2\]
	
% 	If you find this is taking up too much of your precious space, than you are doing something wrong, and it is not adding this little nomenclature.
	
% 	The optional argument specifies the number of column pairs.
	
%   \mininomen[2]{% number of columns
%   $E$ & Energy (\unit{J})                     & $m$ & Mass (\unit{kg}) \\
%   $c$ & Speed of light in vacuum (\unit{m/s}) \\[2ex] % may need some tweaking
%   }
% \end{frame}

\begin{frame}{columns}
  \begin{columns}[onlytextwidth]
    \begin{column}{.5\textwidth}
      first column
    \end{column}
    \begin{column}{.5\textwidth}
      % square filling the column
      \textcolor{tudCyan}{\rule{1\columnwidth}{1\columnwidth}}
      % place an image
      % horizontal position = 73%
      % vertical position = 45%
      % width = 40% of page
      \absimage{.73, .45}{.40}{logo-ugr.pdf}
    \end{column}
  \end{columns}
\end{frame}

\section{Conclusion}
\begin{frame}[fragile]{animation}
  \vfill
  Some commands take optional arguments in the form of \verb|<x-y>|,
  where \verb|x| is the first `sub-frame' on which the context is shown,
  and \verb|y| is the last. \verb|x| or \verb|y| can be replaced by \verb|+|,
  referring to `the next sub-frame'. 
  \vfill
  \begin{columns}[onlytextwidth]
  \begin{column}{.5\textwidth}
    \begin{enumerate}
      \item<+-> uncovered\ldots
      \item<+-> one\ldots
      \item<+-> by\ldots
      \item<+-> one.
    \end{enumerate}
    \end{column}
  \begin{column}{.5\textwidth}
      Using only:\only<1>{1}\only<2>{2}\only<3>{3}

      Using onslide:\onslide<1>{1}\onslide<2>{2}\onslide<3>{3}

      Using pause:\pause1\pause2\pause3
  \end{column}
  \end{columns}
  \vfill
  For more advanced animations, see \S 14 of the manual:\\
  \url{https://www.ctan.org/pkg/beamer}
  % \url{https://www.ctan.org/pkg/animate}\\
  % \url{https://www.ctan.org/pkg/media9}
  \vfill
  % \transduration{2} automatic progression of slides
  \transpush<1>
\end{frame}

\begin{frame}
  Thanks for your attention.

  A digital version of this presentation can be found here:
  \vfill
  \url{https://gitlab.com/novanext/tudelft-beamer} 
  \vfill  
  \centering
  \qrcode{https://gitlab.com/novanext/tudelft-beamer}
  \vfill
\end{frame}


\begin{frame}[allowframebreaks,t]{\bibname}
	% the 'I' is caused by 'allowframebreaks'
	\AtNextBibliography{\footnotesize}% or in the preamble \AtBeginBibliography{\small}
	\printbibliography
\end{frame}


\end{document}

